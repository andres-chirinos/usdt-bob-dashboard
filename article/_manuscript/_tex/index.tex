% Options for packages loaded elsewhere
\PassOptionsToPackage{unicode}{hyperref}
\PassOptionsToPackage{hyphens}{url}
\PassOptionsToPackage{dvipsnames,svgnames,x11names}{xcolor}
%
\documentclass[
  letterpaper,
  DIV=11,
  numbers=noendperiod]{scrartcl}

\usepackage{amsmath,amssymb}
\usepackage{iftex}
\ifPDFTeX
  \usepackage[T1]{fontenc}
  \usepackage[utf8]{inputenc}
  \usepackage{textcomp} % provide euro and other symbols
\else % if luatex or xetex
  \usepackage{unicode-math}
  \defaultfontfeatures{Scale=MatchLowercase}
  \defaultfontfeatures[\rmfamily]{Ligatures=TeX,Scale=1}
\fi
\usepackage{lmodern}
\ifPDFTeX\else  
    % xetex/luatex font selection
\fi
% Use upquote if available, for straight quotes in verbatim environments
\IfFileExists{upquote.sty}{\usepackage{upquote}}{}
\IfFileExists{microtype.sty}{% use microtype if available
  \usepackage[]{microtype}
  \UseMicrotypeSet[protrusion]{basicmath} % disable protrusion for tt fonts
}{}
\makeatletter
\@ifundefined{KOMAClassName}{% if non-KOMA class
  \IfFileExists{parskip.sty}{%
    \usepackage{parskip}
  }{% else
    \setlength{\parindent}{0pt}
    \setlength{\parskip}{6pt plus 2pt minus 1pt}}
}{% if KOMA class
  \KOMAoptions{parskip=half}}
\makeatother
\usepackage{xcolor}
\setlength{\emergencystretch}{3em} % prevent overfull lines
\setcounter{secnumdepth}{1}
% Make \paragraph and \subparagraph free-standing
\makeatletter
\ifx\paragraph\undefined\else
  \let\oldparagraph\paragraph
  \renewcommand{\paragraph}{
    \@ifstar
      \xxxParagraphStar
      \xxxParagraphNoStar
  }
  \newcommand{\xxxParagraphStar}[1]{\oldparagraph*{#1}\mbox{}}
  \newcommand{\xxxParagraphNoStar}[1]{\oldparagraph{#1}\mbox{}}
\fi
\ifx\subparagraph\undefined\else
  \let\oldsubparagraph\subparagraph
  \renewcommand{\subparagraph}{
    \@ifstar
      \xxxSubParagraphStar
      \xxxSubParagraphNoStar
  }
  \newcommand{\xxxSubParagraphStar}[1]{\oldsubparagraph*{#1}\mbox{}}
  \newcommand{\xxxSubParagraphNoStar}[1]{\oldsubparagraph{#1}\mbox{}}
\fi
\makeatother

\usepackage{color}
\usepackage{fancyvrb}
\newcommand{\VerbBar}{|}
\newcommand{\VERB}{\Verb[commandchars=\\\{\}]}
\DefineVerbatimEnvironment{Highlighting}{Verbatim}{commandchars=\\\{\}}
% Add ',fontsize=\small' for more characters per line
\usepackage{framed}
\definecolor{shadecolor}{RGB}{241,243,245}
\newenvironment{Shaded}{\begin{snugshade}}{\end{snugshade}}
\newcommand{\AlertTok}[1]{\textcolor[rgb]{0.68,0.00,0.00}{#1}}
\newcommand{\AnnotationTok}[1]{\textcolor[rgb]{0.37,0.37,0.37}{#1}}
\newcommand{\AttributeTok}[1]{\textcolor[rgb]{0.40,0.45,0.13}{#1}}
\newcommand{\BaseNTok}[1]{\textcolor[rgb]{0.68,0.00,0.00}{#1}}
\newcommand{\BuiltInTok}[1]{\textcolor[rgb]{0.00,0.23,0.31}{#1}}
\newcommand{\CharTok}[1]{\textcolor[rgb]{0.13,0.47,0.30}{#1}}
\newcommand{\CommentTok}[1]{\textcolor[rgb]{0.37,0.37,0.37}{#1}}
\newcommand{\CommentVarTok}[1]{\textcolor[rgb]{0.37,0.37,0.37}{\textit{#1}}}
\newcommand{\ConstantTok}[1]{\textcolor[rgb]{0.56,0.35,0.01}{#1}}
\newcommand{\ControlFlowTok}[1]{\textcolor[rgb]{0.00,0.23,0.31}{\textbf{#1}}}
\newcommand{\DataTypeTok}[1]{\textcolor[rgb]{0.68,0.00,0.00}{#1}}
\newcommand{\DecValTok}[1]{\textcolor[rgb]{0.68,0.00,0.00}{#1}}
\newcommand{\DocumentationTok}[1]{\textcolor[rgb]{0.37,0.37,0.37}{\textit{#1}}}
\newcommand{\ErrorTok}[1]{\textcolor[rgb]{0.68,0.00,0.00}{#1}}
\newcommand{\ExtensionTok}[1]{\textcolor[rgb]{0.00,0.23,0.31}{#1}}
\newcommand{\FloatTok}[1]{\textcolor[rgb]{0.68,0.00,0.00}{#1}}
\newcommand{\FunctionTok}[1]{\textcolor[rgb]{0.28,0.35,0.67}{#1}}
\newcommand{\ImportTok}[1]{\textcolor[rgb]{0.00,0.46,0.62}{#1}}
\newcommand{\InformationTok}[1]{\textcolor[rgb]{0.37,0.37,0.37}{#1}}
\newcommand{\KeywordTok}[1]{\textcolor[rgb]{0.00,0.23,0.31}{\textbf{#1}}}
\newcommand{\NormalTok}[1]{\textcolor[rgb]{0.00,0.23,0.31}{#1}}
\newcommand{\OperatorTok}[1]{\textcolor[rgb]{0.37,0.37,0.37}{#1}}
\newcommand{\OtherTok}[1]{\textcolor[rgb]{0.00,0.23,0.31}{#1}}
\newcommand{\PreprocessorTok}[1]{\textcolor[rgb]{0.68,0.00,0.00}{#1}}
\newcommand{\RegionMarkerTok}[1]{\textcolor[rgb]{0.00,0.23,0.31}{#1}}
\newcommand{\SpecialCharTok}[1]{\textcolor[rgb]{0.37,0.37,0.37}{#1}}
\newcommand{\SpecialStringTok}[1]{\textcolor[rgb]{0.13,0.47,0.30}{#1}}
\newcommand{\StringTok}[1]{\textcolor[rgb]{0.13,0.47,0.30}{#1}}
\newcommand{\VariableTok}[1]{\textcolor[rgb]{0.07,0.07,0.07}{#1}}
\newcommand{\VerbatimStringTok}[1]{\textcolor[rgb]{0.13,0.47,0.30}{#1}}
\newcommand{\WarningTok}[1]{\textcolor[rgb]{0.37,0.37,0.37}{\textit{#1}}}

\providecommand{\tightlist}{%
  \setlength{\itemsep}{0pt}\setlength{\parskip}{0pt}}\usepackage{longtable,booktabs,array}
\usepackage{calc} % for calculating minipage widths
% Correct order of tables after \paragraph or \subparagraph
\usepackage{etoolbox}
\makeatletter
\patchcmd\longtable{\par}{\if@noskipsec\mbox{}\fi\par}{}{}
\makeatother
% Allow footnotes in longtable head/foot
\IfFileExists{footnotehyper.sty}{\usepackage{footnotehyper}}{\usepackage{footnote}}
\makesavenoteenv{longtable}
\usepackage{graphicx}
\makeatletter
\def\maxwidth{\ifdim\Gin@nat@width>\linewidth\linewidth\else\Gin@nat@width\fi}
\def\maxheight{\ifdim\Gin@nat@height>\textheight\textheight\else\Gin@nat@height\fi}
\makeatother
% Scale images if necessary, so that they will not overflow the page
% margins by default, and it is still possible to overwrite the defaults
% using explicit options in \includegraphics[width, height, ...]{}
\setkeys{Gin}{width=\maxwidth,height=\maxheight,keepaspectratio}
% Set default figure placement to htbp
\makeatletter
\def\fps@figure{htbp}
\makeatother
% definitions for citeproc citations
\NewDocumentCommand\citeproctext{}{}
\NewDocumentCommand\citeproc{mm}{%
  \begingroup\def\citeproctext{#2}\cite{#1}\endgroup}
\makeatletter
 % allow citations to break across lines
 \let\@cite@ofmt\@firstofone
 % avoid brackets around text for \cite:
 \def\@biblabel#1{}
 \def\@cite#1#2{{#1\if@tempswa , #2\fi}}
\makeatother
\newlength{\cslhangindent}
\setlength{\cslhangindent}{1.5em}
\newlength{\csllabelwidth}
\setlength{\csllabelwidth}{3em}
\newenvironment{CSLReferences}[2] % #1 hanging-indent, #2 entry-spacing
 {\begin{list}{}{%
  \setlength{\itemindent}{0pt}
  \setlength{\leftmargin}{0pt}
  \setlength{\parsep}{0pt}
  % turn on hanging indent if param 1 is 1
  \ifodd #1
   \setlength{\leftmargin}{\cslhangindent}
   \setlength{\itemindent}{-1\cslhangindent}
  \fi
  % set entry spacing
  \setlength{\itemsep}{#2\baselineskip}}}
 {\end{list}}
\usepackage{calc}
\newcommand{\CSLBlock}[1]{\hfill\break\parbox[t]{\linewidth}{\strut\ignorespaces#1\strut}}
\newcommand{\CSLLeftMargin}[1]{\parbox[t]{\csllabelwidth}{\strut#1\strut}}
\newcommand{\CSLRightInline}[1]{\parbox[t]{\linewidth - \csllabelwidth}{\strut#1\strut}}
\newcommand{\CSLIndent}[1]{\hspace{\cslhangindent}#1}

\KOMAoption{captions}{tableheading}
\makeatletter
\@ifpackageloaded{caption}{}{\usepackage{caption}}
\AtBeginDocument{%
\ifdefined\contentsname
  \renewcommand*\contentsname{Tabla de contenidos}
\else
  \newcommand\contentsname{Tabla de contenidos}
\fi
\ifdefined\listfigurename
  \renewcommand*\listfigurename{Listado de Figuras}
\else
  \newcommand\listfigurename{Listado de Figuras}
\fi
\ifdefined\listtablename
  \renewcommand*\listtablename{Listado de Tablas}
\else
  \newcommand\listtablename{Listado de Tablas}
\fi
\ifdefined\figurename
  \renewcommand*\figurename{Figura}
\else
  \newcommand\figurename{Figura}
\fi
\ifdefined\tablename
  \renewcommand*\tablename{Tabla}
\else
  \newcommand\tablename{Tabla}
\fi
}
\@ifpackageloaded{float}{}{\usepackage{float}}
\floatstyle{ruled}
\@ifundefined{c@chapter}{\newfloat{codelisting}{h}{lop}}{\newfloat{codelisting}{h}{lop}[chapter]}
\floatname{codelisting}{Listado}
\newcommand*\listoflistings{\listof{codelisting}{Listado de Listados}}
\makeatother
\makeatletter
\makeatother
\makeatletter
\@ifpackageloaded{caption}{}{\usepackage{caption}}
\@ifpackageloaded{subcaption}{}{\usepackage{subcaption}}
\makeatother

\ifLuaTeX
\usepackage[bidi=basic]{babel}
\else
\usepackage[bidi=default]{babel}
\fi
\babelprovide[main,import]{spanish}
% get rid of language-specific shorthands (see #6817):
\let\LanguageShortHands\languageshorthands
\def\languageshorthands#1{}
\ifLuaTeX
  \usepackage{selnolig}  % disable illegal ligatures
\fi
\usepackage{bookmark}

\IfFileExists{xurl.sty}{\usepackage{xurl}}{} % add URL line breaks if available
\urlstyle{same} % disable monospaced font for URLs
\hypersetup{
  pdftitle={Historial de Precio del Dólar contra el Boliviano},
  pdflang={es},
  pdfkeywords={Dólar, Boliviano, Criptomonedas, USDT},
  colorlinks=true,
  linkcolor={blue},
  filecolor={Maroon},
  citecolor={Blue},
  urlcolor={Blue},
  pdfcreator={LaTeX via pandoc}}


\title{Historial de Precio del Dólar contra el Boliviano}
\usepackage{etoolbox}
\makeatletter
\providecommand{\subtitle}[1]{% add subtitle to \maketitle
  \apptocmd{\@title}{\par {\large #1 \par}}{}{}
}
\makeatother
\subtitle{Dólar estimado a través del Tether Dollar}
\author{Andres Humberto Chirinos Lizondo \and Kevin Barrientos Carrasco}
\date{}

\begin{document}
\maketitle
\begin{abstract}
El presente trabajo busca almacenar información de las ofertas y el
mercado respecto al precio del Dólar que está evaluado a través de USDT
y otras criptomonedas al Boliviano, siendo un método para encontrar el
valor real del Boliviano en el mercado internacional. Sus resultados se
presentan en un dashboard publicado para estos fines.
\end{abstract}


\section{Introducción}\label{introducciuxf3n}

En los últimos años, las criptomonedas han ganado relevancia en el
ámbito financiero mundial, ofreciendo alternativas a las monedas
tradicionales en transacciones y reservas de valor. El Tether (USDT),
una criptomoneda estable respaldada por el dólar estadounidense, se ha
convertido en una herramienta clave para evaluar el valor de diferentes
monedas en el mercado internacional (Nakamoto 2008).

Bolivia, con su moneda oficial el Boliviano (BOB), mantiene un tipo de
cambio fijo con el dólar estadounidense. Sin embargo, la evaluación del
BOB a través de criptomonedas como el USDT puede ofrecer una perspectiva
diferente sobre su valor real en el mercado internacional. Este enfoque
es particularmente relevante en contextos donde las restricciones
cambiarias y las políticas monetarias pueden distorsionar la percepción
del valor de una moneda (Pérez 2021).

El presente trabajo busca almacenar y analizar información de las
ofertas y el mercado respecto al precio del Dólar, evaluado a través de
USDT y otras criptomonedas populares en Bolivia, principalmente de
múltiples fuentes como Binance. A través de este análisis, se pretende
encontrar el valor real del Boliviano en el mercado internacional y
presentar los resultados en un dashboard interactivo.

\subsection{Objetivo}\label{objetivo}

Determinar el valor real del Boliviano en el mercado internacional
mediante el análisis del historial de precios del USDT frente al BOB, y
presentar los hallazgos en un dashboard interactivo.

\subsection{Hipótesis}\label{hipuxf3tesis}

El tipo de cambio entre el USDT y el BOB refleja de manera más precisa y
dinámica el valor real del Boliviano en el mercado internacional, en
comparación con el tipo de cambio oficial establecido por las
autoridades económicas de Bolivia (García y López 2020).

\section{Materiales y métodos}\label{materiales-y-muxe9todos}

Para llevar a cabo este estudio, se recopilaron datos históricos de las
transacciones de USDT a BOB en plataformas de intercambio de
criptomonedas, principalmente Binance. Los datos incluyen información
sobre el precio, volumen, disponibilidad, tipo de transacción y marcas
de tiempo.

Se utilizó el lenguaje de programación Python y las siguientes
librerías:

\begin{itemize}
\tightlist
\item
  \textbf{pandas}: Para el manejo y procesamiento de datos.
\item
  \textbf{plotly}: Para la generación de gráficos interactivos.
\item
  \textbf{pytz}: Para el manejo de zonas horarias.
\end{itemize}

A continuación, se presenta el fragmento de código utilizado para el
procesamiento de los datos:

\begin{Shaded}
\begin{Highlighting}[]
\ImportTok{import}\NormalTok{ pandas }\ImportTok{as}\NormalTok{ pd}
\ImportTok{import}\NormalTok{ pytz}

\CommentTok{\# Cargar los datos}
\NormalTok{trade\_df }\OperatorTok{=}\NormalTok{ pd.read\_csv(}\StringTok{"data/currency\_exchange\_rates.csv"}\NormalTok{)}

\CommentTok{\# Ajustar la marca de tiempo a la zona horaria GMT{-}4}
\NormalTok{gmt\_minus\_4 }\OperatorTok{=}\NormalTok{ pytz.timezone(}\StringTok{\textquotesingle{}Etc/GMT{-}4\textquotesingle{}}\NormalTok{)}
\NormalTok{trade\_df[}\StringTok{\textquotesingle{}timestamp\textquotesingle{}}\NormalTok{] }\OperatorTok{=}\NormalTok{ pd.to\_datetime(trade\_df[}\StringTok{\textquotesingle{}timestamp\textquotesingle{}}\NormalTok{], unit}\OperatorTok{=}\StringTok{\textquotesingle{}s\textquotesingle{}}\NormalTok{)}\OperatorTok{\textbackslash{}}
\NormalTok{    .dt.tz\_localize(}\StringTok{\textquotesingle{}UTC\textquotesingle{}}\NormalTok{).dt.tz\_convert(gmt\_minus\_4)}

\CommentTok{\# Filtrar y agrupar datos de compra}
\NormalTok{buy\_df }\OperatorTok{=}\NormalTok{ trade\_df.query(}\StringTok{"type == \textquotesingle{}SELL\textquotesingle{}"}\NormalTok{)}
\NormalTok{buy\_df }\OperatorTok{=}\NormalTok{ buy\_df.groupby(buy\_df[}\StringTok{\textquotesingle{}timestamp\textquotesingle{}}\NormalTok{]).agg(}
\NormalTok{    price}\OperatorTok{=}\NormalTok{(}\StringTok{\textquotesingle{}price\textquotesingle{}}\NormalTok{, }\StringTok{\textquotesingle{}min\textquotesingle{}}\NormalTok{),}
\NormalTok{    quantity}\OperatorTok{=}\NormalTok{(}\StringTok{\textquotesingle{}available\textquotesingle{}}\NormalTok{, }\StringTok{\textquotesingle{}sum\textquotesingle{}}\NormalTok{),}
\NormalTok{).reset\_index()}
\end{Highlighting}
\end{Shaded}

Se realizó un agrupamiento basado en la marca de tiempo para obtener los
precios mínimos y máximos de compra y venta, respectivamente, además de
la suma total de la cantidad disponible en cada transacción.

\section{Resultados}\label{resultados}

Los datos procesados permitieron generar gráficos interactivos que
muestran la evolución del precio del USDT frente al BOB a lo largo del
tiempo. A continuación, se presenta el código utilizado para generar el
gráfico principal:

\begin{Shaded}
\begin{Highlighting}[]
\ImportTok{import}\NormalTok{ plotly.express }\ImportTok{as}\NormalTok{ px}

\CommentTok{\# Crear el gráfico de líneas interactivo}
\NormalTok{fig }\OperatorTok{=}\NormalTok{ px.line()}

\CommentTok{\# Añadir datos de compra}
\NormalTok{fig.add\_scatter(x}\OperatorTok{=}\NormalTok{buy\_df[}\StringTok{\textquotesingle{}timestamp\textquotesingle{}}\NormalTok{], y}\OperatorTok{=}\NormalTok{buy\_df[}\StringTok{\textquotesingle{}price\textquotesingle{}}\NormalTok{],}
\NormalTok{                mode}\OperatorTok{=}\StringTok{\textquotesingle{}lines\textquotesingle{}}\NormalTok{, name}\OperatorTok{=}\StringTok{\textquotesingle{}Compra\textquotesingle{}}\NormalTok{, line}\OperatorTok{=}\BuiltInTok{dict}\NormalTok{(color}\OperatorTok{=}\StringTok{\textquotesingle{}blue\textquotesingle{}}\NormalTok{))}

\CommentTok{\# Añadir datos de venta}
\NormalTok{fig.add\_scatter(x}\OperatorTok{=}\NormalTok{sell\_df[}\StringTok{\textquotesingle{}timestamp\textquotesingle{}}\NormalTok{], y}\OperatorTok{=}\NormalTok{sell\_df[}\StringTok{\textquotesingle{}price\textquotesingle{}}\NormalTok{],}
\NormalTok{                mode}\OperatorTok{=}\StringTok{\textquotesingle{}lines\textquotesingle{}}\NormalTok{, name}\OperatorTok{=}\StringTok{\textquotesingle{}Venta\textquotesingle{}}\NormalTok{, line}\OperatorTok{=}\BuiltInTok{dict}\NormalTok{(color}\OperatorTok{=}\StringTok{\textquotesingle{}red\textquotesingle{}}\NormalTok{))}

\CommentTok{\# Añadir línea fija representando el tipo de cambio oficial}
\NormalTok{fig.add\_shape(}
    \BuiltInTok{type}\OperatorTok{=}\StringTok{"line"}\NormalTok{,}
\NormalTok{    x0}\OperatorTok{=}\NormalTok{buy\_df[}\StringTok{\textquotesingle{}timestamp\textquotesingle{}}\NormalTok{].}\BuiltInTok{min}\NormalTok{(),}
\NormalTok{    y0}\OperatorTok{=}\FloatTok{6.96}\NormalTok{,}
\NormalTok{    x1}\OperatorTok{=}\NormalTok{buy\_df[}\StringTok{\textquotesingle{}timestamp\textquotesingle{}}\NormalTok{].}\BuiltInTok{max}\NormalTok{(),}
\NormalTok{    y1}\OperatorTok{=}\FloatTok{6.96}\NormalTok{,}
\NormalTok{    line}\OperatorTok{=}\BuiltInTok{dict}\NormalTok{(color}\OperatorTok{=}\StringTok{"green"}\NormalTok{, width}\OperatorTok{=}\DecValTok{2}\NormalTok{, dash}\OperatorTok{=}\StringTok{"dash"}\NormalTok{),}
\NormalTok{    name}\OperatorTok{=}\StringTok{\textquotesingle{}Tipo de Cambio Oficial\textquotesingle{}}
\NormalTok{)}

\CommentTok{\# Configurar el diseño del gráfico}
\NormalTok{fig.update\_layout(}
\NormalTok{    title}\OperatorTok{=}\StringTok{\textquotesingle{}Historial de Precio del USDT contra el BOB\textquotesingle{}}\NormalTok{,}
\NormalTok{    xaxis\_title}\OperatorTok{=}\StringTok{\textquotesingle{}Fecha y Hora\textquotesingle{}}\NormalTok{,}
\NormalTok{    yaxis\_title}\OperatorTok{=}\StringTok{\textquotesingle{}Precio (BOB)\textquotesingle{}}\NormalTok{,}
\NormalTok{    legend\_title}\OperatorTok{=}\StringTok{\textquotesingle{}Tipo de Comercio\textquotesingle{}}\NormalTok{,}
\NormalTok{    yaxis}\OperatorTok{=}\BuiltInTok{dict}\NormalTok{(tickformat}\OperatorTok{=}\StringTok{\textquotesingle{}.2f\textquotesingle{}}\NormalTok{)}
\NormalTok{)}

\CommentTok{\# Mostrar el gráfico}
\NormalTok{fig.show()}
\end{Highlighting}
\end{Shaded}

El gráfico resultante muestra las fluctuaciones en los precios de compra
y venta del USDT en BOB, permitiendo visualizar tendencias y posibles
discrepancias con el tipo de cambio oficial, representado por la línea
verde discontinua en 6.96.

\section{Discusión}\label{discusiuxf3n}

Los resultados indican que el precio del USDT en BOB presenta
variaciones que pueden reflejar las dinámicas reales del mercado
cambiario en Bolivia. La diferencia entre el tipo de cambio oficial y el
observado en el mercado de criptomonedas sugiere la existencia de
factores económicos y financieros que influyen en la valoración del
Boliviano.

Las posibles causas de estas discrepancias incluyen restricciones en el
acceso a divisas, expectativas inflacionarias, y movimientos
especulativos en el mercado de criptomonedas. Es importante considerar
que, aunque el mercado de criptomonedas ofrece una visión alternativa,
también está sujeto a volatilidades y riesgos inherentes a su naturaleza
descentralizada y menos regulada.

\section{Conclusiones}\label{conclusiones}

El análisis del historial de precios del USDT contra el BOB proporciona
una perspectiva complementaria para evaluar el valor real del Boliviano
en el mercado internacional. Los hallazgos respaldan la hipótesis de que
el tipo de cambio obtenido a través de criptomonedas puede reflejar con
mayor precisión las dinámicas económicas actuales, en comparación con el
tipo de cambio oficial.

La implementación de un dashboard interactivo facilita el acceso a esta
información, permitiendo a investigadores y tomadores de decisiones
analizar en tiempo real las fluctuaciones y tendencias en el mercado
cambiario.

\section*{Agradecimientos}\label{agradecimientos}
\addcontentsline{toc}{section}{Agradecimientos}

Expresamos nuestro agradecimiento a la Carrera de Estadística de la
Universidad Mayor de San Andrés y a la Sociedad Científica de
Estudiantes de Estadística - UMSA por su apoyo y colaboración en esta
investigación.

\section*{Conflicto de intereses}\label{conflicto-de-intereses}
\addcontentsline{toc}{section}{Conflicto de intereses}

Los autores declaran que no existe ningún conflicto de intereses en la
realización de este estudio.

\section*{Referencia Bibliográfica}\label{referencia-bibliogruxe1fica}
\addcontentsline{toc}{section}{Referencia Bibliográfica}

\phantomsection\label{refs}
\begin{CSLReferences}{1}{0}
\bibitem[\citeproctext]{ref-garcia2020analisis}
García, María, y Ricardo López. 2020. {«Análisis del mercado cambiario
boliviano»}. \emph{Anales de Economía Boliviana} 15 (1): 30-45.

\bibitem[\citeproctext]{ref-nakamoto2008bitcoin}
Nakamoto, Satoshi. 2008. {«Bitcoin: A Peer-to-Peer Electronic Cash
System»}. \url{https://bitcoin.org/bitcoin.pdf}.

\bibitem[\citeproctext]{ref-perez2021impacto}
Pérez, Luis. 2021. {«Impacto de las criptomonedas en economías
emergentes»}. \emph{Revista de Economía y Finanzas} 10 (2): 50-65.

\end{CSLReferences}




\end{document}
